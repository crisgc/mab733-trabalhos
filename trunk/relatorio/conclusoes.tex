% -----------------------------------------------------------------------------
% UFRJ
% PPGI
% MAB 733 Sistemas Distribuídos
%
% Revisado em 23/04/2013
% Autor: Cristiano Gurgel de Castro
% -----------------------------------------------------------------------------

\chapter{Conclusões}

Através do presente trabalho foi mostrada a criação de uma aplicação simples
para utilização de \WebService s locais ou disponíveis publicamente.  Algumas
constatações foram tiradas do processo de desenvolvimento:

\begin{itemize}
  \item Nem sempre os serviços disponibilizados gratuitamente na web estarão
    disponíveis. Portanto, aplicações comerciais devem ter muito cuidado ao
    utilizá-los. 
    
    Um outro serviço externo de conversão de temperatura testado tinha
    funcionamento intermitente. 
  \item Deve-se escolher bem as ferramentas para geração de \proxy s para
    utilização de serviços disponibilizados. 
    
    Algumas vezes elas geraram erros
    que tivemos de contornar.
  \item O Código para utilização do \proxy\ gerado automaticamente pode não ser
    nada intuitivo. 
    
    O código gerado pelo \Eclipsev\ é menos intuitivo do gerado
    pelo \NetBeansv. No primeiro, os parâmetros de chamada e o retorno da função
    são encapsulados em classe geradas automaticamente pelo \ApacheAxisDois.
  
  \item O processo de desenvolvimento pode ser bem custoso antes de se ganhar
    experiência na utilização das ferramentas disponíveis e na resolução de
    diversos tipos de problemas.
\end{itemize}
