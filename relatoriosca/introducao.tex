% -----------------------------------------------------------------------------
% UFRJ
% PPGI
% MAB 733 Sistemas Distribuídos
% Trabalho SCA
%
% Revisado em:
% 03/06/2013 por crisgc
% Autor: Cristiano Gurgel de Castro <crisgc1@gmail.com>
% -----------------------------------------------------------------------------


\chapter{Introdução}

No mundo de desenvolvimento de aplicações está-se sempre buscando uma forma
fácil de criar e manter essas aplicações empresariais. Uma alternativa popular para o
desenvolvimento de aplicações empresariais é o SOA, um modelo em que vários
módulos de software são conectados de uma maneira colaborativa em que módulos
proveem serviços, bem como utilizam serviços de outros módulos.

Por em prática a abordagem SOA pode ser difícil, pois é necessário interconectar
módulos os quais foram desenvolvidos utilizando as mais diversas tecnologias.
Para atacar esse problema, o 
\estrangeiro{Service Component Architecture}, ou simplesmente SCA, é proposto como um padrão para
definição de composição (colaboração) entre módulos de software. Plataformas
conhecidas como \estrangeiro{SCA Runtime} são capazes de interpretar as
definições de composição e criar o ambiente necessário para a execução desses
diferentes módulos de acordo com a composição definida. 

Um desses \estrangeiro{SCA Runtime} é o \ApacheTuscany. Essa plataforma,
especificamente, sua versão \ApacheTuscanyV\, foi utilizada nesse trabalho para rodar a
composição definida entre os módulos de software desenvolvidos. O
desenvolvimento desses módulos bem como a composição entre os mesmos são
mostrados no capítulo seguinte.
